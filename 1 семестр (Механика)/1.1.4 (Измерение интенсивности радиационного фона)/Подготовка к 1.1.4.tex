\documentclass[a4paper,12pt]{article} % тип документа

\usepackage{tikz}
\usepackage[T2A]{fontenc}			% кодировка
\usepackage[utf8]{inputenc}			% кодировка исходного текста
\usepackage[english,russian]{babel}	% локализация и переносы
\usepackage{amsfonts,longtable}

% Математика
\usepackage{amsmath,amsfonts,amssymb,amsthm,mathtools} 


\usepackage{wasysym}

\title{Лабораторный журнал к работе 1.1.4 по курсу \\ "Общая физика"  \\ 
\vspace{0.2cm}
\vspace{4.5cm}
 \LARGE{\textbf{Измерение интенсивности радиационного фона}}\vspace{5.5cm}}
\date{28.09.2018}
\usepackage{tikz}
\author{\vspace{0.2cm}Баринов Леонид}

\begin{document}

\maketitle

\newpage

\textbf{Цель работы: } Применение методов обработки экспериментальных данных для изучения статистических закономерностей при измерении радиационного фона

\textbf{Оборуднование:} Счетчик Гейгера-Мюллера (СТС-6), блок питания, компьютер с интерфейсом связи со счетчиком

\textbf{Теоритические данные:} 

В данной работе измеряется число частиц, проходящих через счетчик за 10 и 40 секунд. Выбор времен измерения связан с желанием продемнострировать, что при большем времени лучше выполняется нормальное распределние измеряемых величин и гистограмма более симметрична, чем при малых временах, когда при обработке лучше было бы воспользоваться методами, основанными на другом законе распределения случайных величин, который называется законом Пуассона.

Среднеквадратичная ошибка числа отсчетов, измеренного за некоторый интервал времени, равна корню квадратному из среднего числа отсчетов за тот же интервал: $\sigma = \sqrt{n_0}$. Однако истенное среднее значение измеряемой величины неизвестно. Поэтому в формулу для определения стандартной ошибки отдельного измерения приходится подставлять не истинное среднее значение $n_0$, а измерение значение $n$:
\begin{equation}
\label{1}
\sigma = \sqrt{n}
\end{equation}
Формула (\ref{1}) показывает, что, как правило (с вероятностью $68\%$), измеренное число частиц $n$ отличается от искомого среднего не более чем $\sqrt{n}$. Результат измерений записывается так:
\begin{equation}
\label{2}
n_0 = n \pm \sqrt{n}
\end{equation}
Мы провели серию из $N$ измерений, в результате которых получены числа частиц $n_1, n_2, ..., n_N$. При $N$  измерениях среднее значение числа сосчитанных за одно измерение частиц равно:
\begin{equation}
\label{3}
\overline{n} = \frac{1}{N}\sum_{i=1}^{N}n_i
\end{equation}
Стандартную ошибку отдельного измерения можно оценить по формуле:
\begin{equation}
\label{4}
\sigma_\text{отд} = \sqrt{\frac{1}{N}\sum_{i=1}^{N}(n_i-\overline{n})^2}
\end{equation}
В сооветствии с формулой (\ref{1}) следует ожидать, что эта ошибка будет близка к $\sqrt{n_i}$, т. е. $\sigma_\text{отд}\approx\sigma_i=\sqrt{n_i}$. 
Ближе всего к значению $\sigma_\text{отд}$, определнному по формуле (\ref{4}), лежит, конечно, величина  $\sqrt{\overline{n}}$, т. е.
\begin{equation}
\label{5}
\sigma_\text{отд} \approx \sqrt{\overline{n}}
\end{equation}
Теория вероятностей показывает, что стандартная ошибка отклонения $\overline{n}$ от $n_0$ может быть определена по формуле:
\begin{equation}
\label{6}
\sigma_\text{отд} = \sqrt{\frac{1}{N}\sum_{i=1}^{N}(n_i-\overline{n})^2} = \frac{\sigma_\text{отд}}{\sqrt{N}}
\end{equation}
Для рассмотренной серии из $N$ измерений по 10 с относительная ошибка отдельного измерения (т. е. ожидаемое отличие любого из $n_i$ от $n_0$)
\begin{equation}
\label{7}
\varepsilon_\text{отд} = \frac{\sigma_\text{отд}}{n_i} \approx \frac{1}{\sqrt{n_i}}
\end{equation}
Аналогичным образом определяется относительная ошибка в определении среднего по всем измерениям значения $\overline{n}$:
\begin{equation}
\label{8}
\varepsilon_{\overline{n}} = \frac{\sigma_{\overline{n}}}{\overline{n}} = \frac{\sigma_\text{отд}}{\overline{n} \sqrt{N}} \approx \frac{1}{\sqrt{\overline{n}N}}
\end{equation}
Доля случаев $\omega_n$, характеризующая вероятность получить $n$ отсчетов, определяется по формуле:
\begin{equation}
\label{9}
\omega_n = \frac{\text{число случаев с отсчетом }n}{\text{полное число измерений }(N)}
\end{equation}



\newpage
\textbf{Экспериментальные данные:}

\begin{table}[h]
\centering
\caption{Число срабатываний счетчика за 20с}
\label{table 1}
\renewcommand{\tabcolsep}{4mm}
\begin{tabular}{|c|l|l|l|l|l|l|l|l|l|l|}
\hline
№ опыта & \multicolumn{1}{c|}{1} & \multicolumn{1}{c|}{2} & \multicolumn{1}{c|}{3} & \multicolumn{1}{c|}{4} & \multicolumn{1}{c|}{5} & \multicolumn{1}{c|}{6} & \multicolumn{1}{c|}{7} & \multicolumn{1}{c|}{8} & \multicolumn{1}{c|}{9} & \multicolumn{1}{c|}{10} \\ \hline
0       &                        &                        &                        &                        &                        &                        &                        &                        &                        &                         \\ \hline
10      &                        &                        &                        &                        &                        &                        &                        &                        &                        &                         \\ \hline
20      &                        &                        &                        &                        &                        &                        &                        &                        &                        &                         \\ \hline
30      &                        &                        &                        &                        &                        &                        &                        &                        &                        &                         \\ \hline
40      &                        &                        &                        &                        &                        &                        &                        &                        &                        &                         \\ \hline
50      &                        &                        &                        &                        &                        &                        &                        &                        &                        &                         \\ \hline
60      &                        &                        &                        &                        &                        &                        &                        &                        &                        &                         \\ \hline
70      &                        &                        &                        &                        &                        &                        &                        &                        &                        &                         \\ \hline
80      &                        &                        &                        &                        &                        &                        &                        &                        &                        &                         \\ \hline
90      &                        &                        &                        &                        &                        &                        &                        &                        &                        &                         \\ \hline
100     &                        &                        &                        &                        &                        &                        &                        &                        &                        &                         \\ \hline
110     &                        &                        &                        &                        &                        &                        &                        &                        &                        &                         \\ \hline
120     &                        &                        &                        &                        &                        &                        &                        &                        &                        &                         \\ \hline
130     &                        &                        &                        &                        &                        &                        &                        &                        &                        &                         \\ \hline
140     &                        &                        &                        &                        &                        &                        &                        &                        &                        &                         \\ \hline
150     &                        &                        &                        &                        &                        &                        &                        &                        &                        &                         \\ \hline
160     &                        &                        &                        &                        &                        &                        &                        &                        &                        &                         \\ \hline
170     &                        &                        &                        &                        &                        &                        &                        &                        &                        &                         \\ \hline
180     &                        &                        &                        &                        &                        &                        &                        &                        &                        &                         \\ \hline
190     &                        &                        &                        &                        &                        &                        &                        &                        &                        &                         \\ \hline
\end{tabular}
\end{table}

\begin{table}[h!]
\renewcommand{\tabcolsep}{5mm}
\centering
\caption{Данные для построения гистограммы распределения числа срабатываний счетчика за 10с}
\label{table 2}
\begin{tabular}{|c|c|c|c|c|c|c|}
\hline
Число импульсов $n_i$ & 0 & 1 & 2 & 3 & 4 & 5 \\ \hline
Число случаев   &   &   &   &   &   &   \\ \hline
Доля случаев $\omega_n$   &   &   &   &   &   &   \\ \hline\hline
Число импульсов $n_i$ & 6 & 7 & 8 & 9 &10 & 11 \\ \hline
Число случаев   &   &   &   &   &   &   \\ \hline
Доля случаев $\omega_n$   &   &   &   &   &   &   \\ \hline\hline
Число импульсов $n_i$ & 12 & 13 & 14 & 15 &16 & 17 \\ \hline
Число случаев   &   &   &   &   &   &   \\ \hline
Доля случаев $\omega_n$   &   &   &   &   &   &   \\ \hline
\end{tabular}
\end{table}

\newpage
\begin{table}[h]
\centering
\caption{Число срабатываний счетчика за 40с}
\label{table 3}
\renewcommand{\tabcolsep}{4mm}
\begin{tabular}{|c|l|l|l|l|l|l|l|l|l|l|}
\hline
№ опыта & \multicolumn{1}{c|}{1} & \multicolumn{1}{c|}{2} & \multicolumn{1}{c|}{3} & \multicolumn{1}{c|}{4} & \multicolumn{1}{c|}{5} & \multicolumn{1}{c|}{6} & \multicolumn{1}{c|}{7} & \multicolumn{1}{c|}{8} & \multicolumn{1}{c|}{9} & \multicolumn{1}{c|}{10} \\ \hline
0       &                        &                        &                        &                        &                        &                        &                        &                        &                        &                         \\ \hline
10      &                        &                        &                        &                        &                        &                        &                        &                        &                        &                         \\ \hline
20      &                        &                        &                        &                        &                        &                        &                        &                        &                        &                         \\ \hline
30      &                        &                        &                        &                        &                        &                        &                        &                        &                        &                         \\ \hline
40      &                        &                        &                        &                        &                        &                        &                        &                        &                        &                         \\ \hline
50      &                        &                        &                        &                        &                        &                        &                        &                        &                        &                         \\ \hline
60      &                        &                        &                        &                        &                        &                        &                        &                        &                        &                         \\ \hline
70      &                        &                        &                        &                        &                        &                        &                        &                        &                        &                         \\ \hline
80      &                        &                        &                        &                        &                        &                        &                        &                        &                        &                         \\ \hline
90      &                        &                        &                        &                        &                        &                        &                        &                        &                        &                         \\ \hline
\end{tabular}
\end{table}

\begin{table}[h!]
\renewcommand{\tabcolsep}{3mm}
\centering
\caption{Данные для построения гистограммы распределения числа срабатываний счетчика за 40с}
\label{table 4}
\begin{tabular}{|c|c|c|c|c|c|c|c|c|c|}
\hline
Число импульсов $n_i$ & 17 & 18 & 19 & 20 & 21 & 22  & 23 & 24 & 25\\ \hline
Число случаев   &   &   &   &   &   &  &	&	& \\ \hline
Доля случаев $\omega_n$   &   &   &   &   &   &  &	&	&  \\ \hline\hline
Число импульсов $n_i$ & 26 &27 & 28 & 29 & 30 & 31 & 32 &33 & 34\\ \hline
Число случаев   &   &   &   &   &   &  &	&	&   \\ \hline
Доля случаев $\omega_n$   &   &   &   &   &   &  &	&	&   \\ \hline\hline
Число импульсов $n_i$ & 35 &36 & 37 & 38 & 39 & 40 & 41 &42 & 43\\ \hline
Число случаев   &   &   &   &   &   &  &	&	&   \\ \hline
Доля случаев $\omega_n$   &   &   &   &   &   &  &	&	&   \\ \hline
\end{tabular}
\end{table}

\begin{table}[h!]
\renewcommand{\tabcolsep}{3mm}
\centering
\caption{Сравнение теоритической и эксперементальной доли случаев}
\label{table 5}
\begin{tabular}{|p{3cm}|c|c|p{3cm}|}
\hline
Ошибка & Число случаев & Доля случаев, & Теоретическая оценка\\ \hline
$\pm\sigma_1= $ & & &\\ \hline
$\pm2\sigma_1= $ & & &\\ \hline
\end{tabular}
\end{table}



\end{document}