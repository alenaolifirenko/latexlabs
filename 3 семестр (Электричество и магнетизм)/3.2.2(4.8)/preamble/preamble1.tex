%--------------------------------------------------------------------
% Эта преамбула с комментариями для написания лабораторных работ по
% физике. В его основе информация из книги С. М. Львовского "Набор и
% верстка в пакете Latex", а также материалы по курсу "Документы и
% презентации в Latex" от ВШЭ https://www.coursera.org/learn/latex. Ну
% и мой опыт (1 год и 15 лабораторных работ + 2 Вопроса по выбору)
% Автор - Баринов Леонид
% Дата - 06.08.2019
%--------------------------------------------------------------------
%--------------------------------------------------------------------
% Для начала необходимо определиться с типом документа. Оптимальный
% (на мой взгляд) вариант - article. Также существуют типы book,
% report, proc и другие. Также в необязательном аргументе можно
% указать тип страницы и размер шрифта. Стандарт по умолчанию - А4 и
% 12 (иногда 10) шрифт. Необязательный аргумент шрифта может принимать
% только 3 параметра - 10, 11, 12 (pt).

\documentclass[a4paper, 12pt]{article}

%--------------------------------------------------------------------
% Чтобы использовать другие размеры шрифта используется пакет
% extsizes. Он позволяет указывать в \documentclass такие размеры - 8,
% 9, 10, 11, 12, 14, 17, 20 (pt). При указании других размеров могут
% возникать различные проблемы.

\usepackage{extsizes}

%--------------------------------------------------------------------
% Необходимо определиться с кодировкой документа. Идеального варианта
% для русского языка не существует - каждый чем-то немного плох. Для
% особо интересующихся - Приложение И в 5 издании книги Львовского. Я
% воспользовался вариантом, предлагаемым на курсе по Latex от ВШЭ.

\usepackage[T2A]{fontenc}
\usepackage[utf8]{inputenc}

%--------------------------------------------------------------------
% Для соблюдения типографских традиций (оказывается такие существуют)
% различных стран создан пакет babel. Самое заметное его действие -
% latex научиться переносить слова того языка, который вы укажите.
% Можно указать несколько языков через запятую. Основной язык
% документа указывается последним.

\usepackage[english,russian]{babel}

%--------------------------------------------------------------------
% Перейдем к заданию полей документа. Есть несколько способов, но
% самый простой из них - это воспользоваться пакетом geometry, который
% позволяет определить все поля документа (начиная с краев листа, что
% важно, так как некоторые другие способы позволяют это сделать только
% косвенно)

\usepackage{geometry}
\geometry{top=25mm}
\geometry{bottom=35mm}
\geometry{left=35mm}
\geometry{right=20mm}

%--------------------------------------------------------------------
% От полей логично перейти к колонтитулам. Тут нам поможет пакет
% fancyhdr. Для него существует 6 колонтитулов - верхний, левый;
% верхний, по центру; верхний, правый и такие же нижние. По умолчанию
% номер страницы находится снизу по центру, а также существует
% линейка, очерчивающие верхний колонтитул. Мне показалось интересным
% сделать колонтитулы схожие с колонтитулами в лабнике. 

\usepackage{fancyhdr}
\pagestyle{fancy}
\renewcommand{\sectionmark}[1]{\markboth{#1}{}} 
% \renewcommand{\headrulewidth}{0mm} % Если необходимо убрать линейку,
% или изменить ее длину
% \lfoot{} % Нижний левый
% \rfoot{} % Нижний правый
% \rhead{} % Верхний правый
% \chead{} % Верхний в центре
\lhead{\thepage} % Номер страницы в левом верхнем углу
\cfoot{} % Оставить нижний колонтитул без цифры

%--------------------------------------------------------------------
% Самое время научиться работать с формулами. А точнее добавить пакеты
% от Американского математического общества, которые позволять
% пользоваться большим количеством математических символов.

\usepackage{amsmath,amsfonts,amssymb,amsthm,mathtools}

%--------------------------------------------------------------------
% Также очень хочется пользоваться русскими буквами в формулах, для
% этого подключаем пакет mathtext, который добавляет окружение
% \text{}. Внутри него можно писать русские буквы в математическом
% режиме.

\usepackage{mathtext}

%--------------------------------------------------------------------
% Большим преимуществом вашего pdf документа будет возможность поиска
% в нем по словам или буквами. (Например, в Ивановнике это
% невозможно)

\usepackage{cmap}

%--------------------------------------------------------------------
% Куда же в физике без картинок и графиков? Давайте исправим
% эту недоработку

\usepackage{graphicx}
\graphicspath{{images/}} % Необходимо, если рисунки
% находятся в другой папке

%--------------------------------------------------------------------
% graphicx не позволяет вставлять обтекаемые рисунки, но на
% практике они очень нужны. Для этого существует пакет wrapfig

\usepackage{wrapfig}

%--------------------------------------------------------------------
% latex вставляет рисунки по определенному алгоритму. Его,
% конечно, можно менять, но это не настолько просто. Как
% правило, хочется, чтобы картинка располагалась там, где мы это
% указали в коде. Для этого существует несколько пакетов, один из
% них floatrow. Он позволяет для окружения figure указывать
% необязательный аргумент - H (именно большое h), что на latex'овском
% языке означает: вставить картинку здесь и только здесь. (даже если
% облик документа несколько пострадает)

\usepackage{floatrow}

%--------------------------------------------------------------------
% По правилам оформления рисунок всегда должен быть подписан. Для
% этого существует команда \caption{}. Но обычные настройки caption
% меня не совсем устроили. Хотелось сделать подпись меньше
% основного шрифта, а также слово Рис жирным и использовать
% разделитель точку, а не двоеточие. В этом помогает пакет,
% который называется caption (совпадение?)

\usepackage[margin=10pt,font=small,labelfont=bf,labelsep=period]{caption}

%--------------------------------------------------------------------
% Последним важным пунктом остались таблицы. Ведь куда-то нужно
% заносить результаты измерений. На данный момент во время выполнения
% лабораторных работ я заношу результаты в таблицу excel, а потом с
% помощью сайта www.tablesgenerator.com превращаю в таблицу latex и
% дооформляю.

\usepackage{array,tabularx,tabulary,booktabs}

%--------------------------------------------------------------------
% После excel есть ощущения, что везде объединить колонки или строки
% легко. В latex не совсем так. Помогают пакеты multirow, multicol. 

\usepackage{multirow}
\usepackage{multicol}

%--------------------------------------------------------------------
% Иногда могут потребоваться длинные таблицы на несколько страниц.
% Обычные таблицы latex воспринимает как одну букву. И
% становиться понятно, почему возникают проблемы при переносе
% обычной таблицы. (Ведь нельзя же перенести одну букву!). Поэтому
% вместо обычной таблицы нужна длинная таблица.

\usepackage{longtable}

%--------------------------------------------------------------------
% Часто в таблице хочется сделать перенос текста или формулы. Просто
% так это сделать не получиться из-за синтаксиса tabular. Для этого
% каждый раз необходимо создавать новое окружение tabular, что
% утомительно. Поэтому можно ввести команду \specialcell
% (назвать можно по-любому)
    
%\newcommand{\specialcell}[2][c]{%
%	\begin{tabular}[#1]{@{}c{}}#2\end{tabular}}

%--------------------------------------------------------------------
% Когда в таблице много колонок и строк, кажется, что они находятся
% слишком близко к друг другу. Можно переопределить
% несколько параметров, чтобы выглядело лучше. Это можно сделать либо
% в преамбуле, либо непосредственно в документе. Первое
% переопределение отвечает за интервал между строками, второе за
% интервал между колонками

% \renewcommand{\arraystretch}{1.8} 
% \renewcommand{\tabcolsep}{1cm} 

%--------------------------------------------------------------------
% В русской типографской традиции принято начинать каждый новый абзац
% с красной строки. Даже первый после заголовка (или подзаголовка).
% Чтобы каждый раз не ставить красную строку вручную существует пакет
% indentfirst

\usepackage{indentfirst}

%--------------------------------------------------------------------
% Некоторые модификаторы начертания

\usepackage{soul}
\usepackage{soulutf8}

\usepackage{mathabx}
