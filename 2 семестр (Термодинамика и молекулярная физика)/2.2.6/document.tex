% Этот шаблон документа разработан в 2014 году
% Данилом Фёдоровых (danil@fedorovykh.ru) 
% для использования в курсе 
% <<Документы и презентации в \LaTeX>>, записанном НИУ ВШЭ
% для Coursera.org: http://coursera.org/course/latex .
% Исходная версия шаблона --- 
% https://www.writelatex.com/coursera/latex/3.2

\documentclass[a4paper,12pt]{article}

%%% Работа с русским языком
\usepackage{cmap}					% поиск в PDF
\usepackage{mathtext} 				% русские буквы в формулах
\usepackage[T2A]{fontenc}			% кодировка
\usepackage[utf8]{inputenc}			% кодировка исходного текста
\usepackage[english,russian]{babel}	% локализация и переносы

%%% Дополнительная работа с математикой
\usepackage{amsmath,amsfonts,amssymb,amsthm,mathtools} % AMS
\usepackage{icomma} % "Умная" запятая: $0,2$ --- число, $0, 2$ --- перечисление

%% Номера формул
%\mathtoolsset{showonlyrefs=true} % Показывать номера только у тех формул, на которые есть \eqref{} в тексте.
%\usepackage{leqno} % Нумерация формул слева

%% Свои команды
\DeclareMathOperator{\sgn}{\mathop{sgn}}

%% Перенос знаков в формулах (по Львовскому)
\newcommand*{\hm}[1]{#1\nobreak\discretionary{}
	{\hbox{$\mathsurround=0pt #1$}}{}}

%%% Работа с картинками
\usepackage{graphicx}  % Для вставки рисунков
\graphicspath{{images/}{images2/}}  % папки с картинками
\setlength\fboxsep{3pt} % Отступ рамки \fbox{} от рисунка
\setlength\fboxrule{1pt} % Толщина линий рамки \fbox{}
\usepackage{wrapfig} % Обтекание рисунков текстом

%%% Работа с таблицами
\usepackage{array,tabularx,tabulary,booktabs} % Дополнительная работа с таблицами
\usepackage{longtable}  % Длинные таблицы
\usepackage{multirow} % Слияние строк в таблице

%%% Теоремы
\theoremstyle{plain} % Это стиль по умолчанию, его можно не переопределять.
\newtheorem{theorem}{Теорема}[section]
\newtheorem{proposition}[theorem]{Утверждение}

\theoremstyle{definition} % "Определение"
\newtheorem{corollary}{Следствие}[theorem]
\newtheorem{problem}{Задача}[section]

\theoremstyle{remark} % "Примечание"
\newtheorem*{nonum}{Решение}

%%% Программирование
\usepackage{etoolbox} % логические операторы

%%% Страница
\usepackage{extsizes} % Возможность сделать 14-й шрифт
\usepackage{geometry} % Простой способ задавать поля
\geometry{top=25mm}
\geometry{bottom=35mm}
\geometry{left=35mm}
\geometry{right=20mm}
%
\usepackage{fancyhdr} % Колонтитулы
\pagestyle{fancy}
\renewcommand{\headrulewidth}{0mm}  % Толщина линейки, отчеркивающей верхний колонтитул
%\lfoot{Нижний левый}
%\rfoot{Нижний правый}
\rhead{}
%\chead{Верхний в центре}
%\lhead{Верхний левый}
% \cfoot{Нижний в центре} % По умолчанию здесь номер страницы

\usepackage{setspace} % Интерлиньяж
%\onehalfspacing % Интерлиньяж 1.5
%\doublespacing % Интерлиньяж 2
%\singlespacing % Интерлиньяж 1

\usepackage{lastpage} % Узнать, сколько всего страниц в документе.

\usepackage{soul} % Модификаторы начертания

\usepackage{indentfirst} % Красная строка

\usepackage{soulutf8} % Модификаторы начертания

\usepackage{hyperref}
\usepackage[usenames,dvipsnames,svgnames,table,rgb]{xcolor}
\hypersetup{				% Гиперссылки
	unicode=true,           % русские буквы в раздела PDF
	pdftitle={Заголовок},   % Заголовок
	pdfauthor={Автор},      % Автор
	pdfsubject={Тема},      % Тема
	pdfcreator={Создатель}, % Создатель
	pdfproducer={Производитель}, % Производитель
	pdfkeywords={keyword1} {key2} {key3}, % Ключевые слова
	colorlinks=true,       	% false: ссылки в рамках; true: цветные ссылки
	linkcolor=red,          % внутренние ссылки
	citecolor=green,        % на библиографию
	filecolor=magenta,      % на файлы
	urlcolor=cyan           % на URL
}

%\renewcommand{\familydefault}{\sfdefault} % Начертание шрифта

\usepackage{multicol} % Несколько колонок

\author{\LaTeX{} в Вышке}
\title{3.2 Оформление документа в целом}
\date{\today}

\begin{document} % конец преамбулы, начало документа
\thispagestyle{empty}
\begin{center}
	\textit{Федеральное государственное автономное образовательное\\ учреждение высшего образования }
	\vspace{0.5ex}
	
	\textbf{«Московский физико-технический институт\\ (национальный исследовательский университет)»}
\end{center}
\vspace{10ex}
%\begin{flushright}
%	\noindent
%	\textit{Фамилия Имя Отчество}
%	\\
%	\textit{студент факультета экономики \\(группа 211И)}
%\end{flushright}
\begin{center}
	\vspace{13ex}
	\so{\textbf{Лабораторная работа №1.3.3}}
	\vspace{1ex}
	
	по курсу общей физики
	
	
	на тему:
	
	\textbf{\textit{<<Определение вязкости воздуха\\ по скорости течения через тонкие трубки>>}}
	\vspace{30ex}
	\begin{flushright}
		\noindent
		\textit{Работу выполнил:}
		\\
		\textit{Баринов Леонид \\(группа Б02-827)}
	\end{flushright}
	\vfill
	Долгопрудный \\2019 год
\end{center}

\newpage
\setcounter{page}{1}
\section{Аннотация}
В работе будет экспериментально выявлен участок сформированного течения, определены режимы ламинарного и турбулентного течения, определено число Рейнольдса.
\section{Теоритические сведения}
\subsection{Формула Пуазейля}
Уравнение Бернулли:
\begin{equation}
\frac{\upsilon^2}{2}+gh_1+\frac{p}{\rho} = const
\end{equation}

В соответствии с уравнением Бернулли при стационарном течении по прямолинейной горизонтальной трубе постоянного сечения давление жидкости должно быть одним и тем же по всей длине трубы. В действительности, однако, давление жидкости в трубе падает в направлении течения. Для обеспечения стационарности течения необходимо поддерживать на концах трубы постоянную разность давлений, уравновешивающую силы внутреннего трения, которые возникают при течении жидкости.

Сила трения между слоями жидкости зависит от изменения скорости в перпендикулярном потоку направлении (закон Ньютона для вязкой жидкости):
\begin{equation}
F = S\eta\frac{d\upsilon_x}{dy}
\end{equation}

Для вывода формулу Пуазейля рассмотрим следующую физическую модель. Пусть вязкая несжимаемая жидкость течет вдоль прямолинейной цилиндрической трубы радиусом $R$. Координатную ось $x$ направим вдоль оси трубы в сторону течения
\end{document} % конец документа

	